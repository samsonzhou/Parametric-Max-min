\section{Introduction}

Parametric search is a powerful technique for solving various
optimization problems
\cite{AP,C1,C2,CSSS,FJ1,FJ2,Gu,M,R,S,Z}.
To solve a given problem by parametric search, one must find a threshold value, namely a largest (or smallest) value
that passes a certain feasibility test.
In general, we must identify explicitly or implicitly a set of candidate values, and search the set, 
choosing one value at a time upon which to test feasibility.
As a result of the test, we discard from further consideration various values in the set.
Eventually, the search will converge on the optimal value.
 
In \cite{M}, Megiddo emphasized the role that
parallel algorithms can play in the implementation of such
a technique.
In \cite{C2}, Cole improved the time bounds for a number of problems
addressed in \cite{M},
abstracting two techniques upon which parametric search can be based:
sorting and performing independent binary searches.
Despite the clever ideas elaborated in these papers,
none of the algorithms presented in \cite{M, C2} 
is known to be optimal.
In fact, rarely have algorithms for parametric search problems
been shown to be optimal.
(For one that is optimal, see \cite{CSSS}.)
In at least some cases, the time required for feasibility testing
matches known lower bounds for the parametric search problem.
But in the worst case, $\Omega (\log n)$ values must be tested for feasibility, where $n$ is the size of the input.
Is this extra factor of at least $\log n$ necessary?
For several parametric search problems on paths and trees,
we show that a polylogarithmic penalty in the running time can be avoided,
and give linear-time (and hence optimal) algorithms for these problems.

We consider the max-min tree $k$-partitioning problem \cite{PS}.
Let $T$ be a tree with $n$ vertices and a nonnegative weight associated with each vertex. 
Let $k < n$ be a given positive integer.
The problem is to delete $k$ edges in the tree so as to maximize the weight of the lightest of the resulting subtrees.
Perl and Schach introduced an algorithm that runs in $O(k^2 rd(T)+kn)$ time, where $rd(T)$ is the radius of the tree \cite{PS}.
Megiddo and Cole first considered the special case in which the tree is a path, and gave $O(n (\log n)^2)$-time and $O(n \log n)$-time algorithms, resp., for this case.\footnote{All logarithms are to the base 2.}
(Megiddo noted that an $O(n \log n)$-time algorithm is possible, using ideas from \cite{FJ1}.)
For the problem on a tree, Megiddo presented an $O(n( \log n)^3)$-time algorithm \cite{M}, and Cole presented an $O(n( \log n)^2)$-time algorithm \cite{C2}. 
The algorithm we present here runs in $O(n)$ time, which is clearly optimal.

A closely related problem is the min-max path $k$-partitioning problem \cite{BPS},  for which we must delete $k$ edges in the path so as to minimize the weight of the heaviest of the resulting subpaths. 
Our techniques yield linear-time algorithms for both versions of path-partitioning  and for max-min tree $k$-partitioning problem. Our approach is less convoluted than that in \cite{FTR, F1}, and thus its analysis should be more amenable to independent confirmation and publication. 
We believe that this paper provides for the full acknowledgement of linear time  algorithms for these particular problems.

Our results are a contribution to parametric search in several ways
beyond merely producing optimal algorithms for path and tree partitioning.  
In contrast to \cite{F1}, we introduce synthetic weights on candidate values 
or groups of candidate values from a certain implicit matrix. 
We use weighted selection to resolve the values representing many shorter paths before the values for longer paths, enabling future feasibility tests to run faster. 
We also use unweighted selection to reduce the size of the set of candidate values so that the time for selecting 
test values does not become an obstacle to achieving linear time. 

Our parametric search on trees reduces feasibility test time across subpaths, while simultaneously also pruning paths that
have no paths that branch off of them. To demonstrate progress, we identify an effective measure of current problem size. 
Surprisingly, the number of vertices remaining in the tree seems not to be so helpful, as is also the case for the number of candidate values. 
Instead, we show that the time to perform a feasibility test actually is effective. 
Our analysis for tree partitioning captures the total progress from the beginning of the algorithm, not necessarily between consecutive rounds, as was the case for the path-partitioning problem. 
Thus it seems necessary to have a dual-pronged strategy to simultaneously compress the search structure of paths and also prune paths that end with a leaf.

We organize our paper as follows. 
In Section 2 we discuss features of parametric search, and lay the foundation for the optimal algorithms that we describe in subsequent sections.
In Section 3 we present our approach for solving both
the min-max and the max-min partitioning problem on a path. 
In Section 4 we build on the results in Section 3 and
present our approach for optimally solving the max-min partitioning problem on a tree.
\bigskip

