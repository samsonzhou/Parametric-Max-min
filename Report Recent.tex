\documentclass[12pt]{article}
\newtheorem{define}{Definition}
%\newcommand{\F}{{\mathbb{F}}}
%\usepackage{psfig}
\usepackage{amsmath,amssymb,amsfonts}
\usepackage[dvips]{graphics,color}
\usepackage{latexsym}
\usepackage{setspace}
\usepackage{color}
%\usepackage{fullpage}

\oddsidemargin=0.15in
\evensidemargin=0.15in
\topmargin=-.5in
\textheight=9in
\textwidth=6.25in

\begin{document}
\newtheorem{theorem}{Theorem}[section]
\newtheorem{corollary}[theorem]{Corollary}
\newtheorem{lemma}[theorem]{Lemma}
\newtheorem{observation}[theorem]{Observation}
\newtheorem{proposition}[theorem]{Proposition}
\newtheorem{definition}[theorem]{Definition}
\newtheorem{claim}[theorem]{Claim}
\newtheorem{fact}[theorem]{Fact}
\newtheorem{assumption}[theorem]{Assumption}
%\pgfplotsset{compat=1.5}

\newcommand{\qed}{\rule{7pt}{7pt}}
\newcommand{\dis}{\mathop{\mbox{\rm d}}\nolimits}
\newcommand{\per}{\mathop{\mbox{\rm per}}\nolimits}
\newcommand{\area}{\mathop{\mbox{\rm area}}\nolimits}
\newcommand{\cw}{\mathop{\rm cw}\nolimits}
\newcommand{\ccw}{\mathop{\rm ccw}\nolimits}
\newcommand{\DIST}{\mathop{\mbox{\rm DIST}}\nolimits}
\newcommand{\OP}{\mathop{\mbox{\it OP}}\nolimits}
\newcommand{\OPprime}{\mathop{\mbox{\it OP}^{\,\prime}}\nolimits}
\newcommand{\ihat}{\hat{\imath}}
\newcommand{\jhat}{\hat{\jmath}}
\newcommand{\abs}[1]{\mathify{\left| #1 \right|}}

\newenvironment{proof}{\noindent{\bf Proof}\hspace*{1em}}{\qed\bigskip}
\newenvironment{proof-sketch}{\noindent{\bf Sketch of Proof}\hspace*{1em}}{\qed\bigskip}
\newenvironment{proof-idea}{\noindent{\bf Proof Idea}\hspace*{1em}}{\qed\bigskip}
\newenvironment{proof-of-lemma}[1]{\noindent{\bf Proof of Lemma #1}\hspace*{1em}}{\qed\bigskip}
\newenvironment{proof-attempt}{\noindent{\bf Proof Attempt}\hspace*{1em}}{\qed\bigskip}
\newenvironment{proofof}[1]{\noindent{\bf Proof}
of #1:\hspace*{1em}}{\qed\bigskip}
\newenvironment{remark}{\noindent{\bf Remark}\hspace*{1em}}{\bigskip}

% \makeatletter
% \@addtoreset{figure}{section}
% \@addtoreset{table}{section}
% \@addtoreset{equation}{section}
% \makeatother

\newcommand{\FOR}{{\bf for}}
\newcommand{\TO}{{\bf to}}
\newcommand{\DO}{{\bf do}}
\newcommand{\REPEAT}{{\bf repeat}}
\newcommand{\UNTIL}{{\bf until}}
\newcommand{\WHILE}{{\bf while}}
\newcommand{\AND}{{\bf and}}
\newcommand{\IF}{{\bf if}}
\newcommand{\THEN}{{\bf then}}
\newcommand{\ELSE}{{\bf else}}

% \renewcommand{\thefigure}{\thesection.\arabic{figure}}
% \renewcommand{\thetable}{\thesection.\arabic{table}}
% \renewcommand{\theequation}{\thesection.\arabic{equation}}

\makeatletter
\def\fnum@figure{{\bf Figure \thefigure}}
\def\fnum@table{{\bf Table \thetable}}
\long\def\@mycaption#1[#2]#3{\addcontentsline{\csname
  ext@#1\endcsname}{#1}{\protect\numberline{\csname 
  the#1\endcsname}{\ignorespaces #2}}\par
  \begingroup
    \@parboxrestore
    \small
    \@makecaption{\csname fnum@#1\endcsname}{\ignorespaces #3}\par
  \endgroup}
\def\mycaption{\refstepcounter\@captype \@dblarg{\@mycaption\@captype}}
\makeatother

\newcommand{\figcaption}[1]{\mycaption[]{#1}}
\newcommand{\tabcaption}[1]{\mycaption[]{#1}}
\newcommand{\head}[1]{\chapter[Lecture \##1]{}}
\newcommand{\mathify}[1]{\ifmmode{#1}\else\mbox{$#1$}\fi}
%\renewcommand{\Pr}[1]{\mathify{\mbox{Pr}\left[#1\right]}}
%\newcommand{\Exp}[1]{\mathify{\mbox{Exp}\left[#1\right]}}
\newcommand{\bigO}O
\newcommand{\set}[1]{\mathify{\left\{ #1 \right\}}}
\def\half{\frac{1}{2}}

% Coding theory addenda

\newcommand{\enc}{{\sf Enc}}
\newcommand{\dec}{{\sf Dec}}
\newcommand{\E}{{\rm Exp}}
\newcommand{\Var}{{\rm Var}}
\newcommand{\Z}{{\mathbb Z}}
\newcommand{\F}{{\mathbb F}}
\newcommand{\integers}{{\mathbb Z}^{\geq 0}}
\newcommand{\R}{{\mathbb R}}
\newcommand{\Q}{{\cal Q}}
\newcommand{\eqdef}{{\stackrel{\rm def}{=}}}
\newcommand{\from}{{\leftarrow}}
\newcommand{\vol}{{\rm Vol}}
\newcommand{\poly}{{\rm poly}}
\newcommand{\ip}[1]{{\langle #1 \rangle}}
\newcommand{\wt}{{\rm wt}}
\renewcommand{\vec}[1]{{\mathbf #1}}
\newcommand{\mspan}{{\rm span}}
\newcommand{\rs}{{\rm RS}}
\newcommand{\RM}{{\rm RM}}
\newcommand{\Had}{{\rm Had}}
\newcommand{\calc}{{\cal C}}


\newcommand{\fig}[4]{
        \begin{figure}
        \setlength{\epsfysize}{#2}
        \vspace{3mm}
        \centerline{\epsfbox{#4}}
        \caption{#3} \label{#1}
        \end{figure}
        }

\newcommand{\ord}{{\rm ord}}

\providecommand{\norm}[1]{\lVert #1 \rVert}
\newcommand{\embed}{{\rm Embed}}
\newcommand{\qembed}{\mbox{$q$-Embed}}
\newcommand{\calh}{{\cal H}}
\newcommand{\lp}{{\rm LP}}


\newcommand{\sspace}{\baselineskip 14pt}
\newcommand{\dspace}{\baselineskip 24pt}
%\newcommand{\dspace}{\baselineskip 18pt}
\newcommand{\ASG}{\leftarrow}
\newcommand{\TN}{{\bf then}}
\newcommand{\EL}{{\bf else}}
\newcommand{\EI}{{\bf endif}}
\newcommand{\RE}{{\bf repeat}}
\newcommand{\UN}{{\bf until}}
\newcommand{\ER}{{\bf endrepeat}}
\newcommand{\WH}{{\bf while}}
\newcommand{\EW}{{\bf endwhile}}
\newcommand{\FO}{{\bf for}}
\newcommand{\BY}{{\bf by}}
\newcommand{\EF}{{\bf endfor}}
\newcommand{\CA}{{\bf case}}
\newcommand{\EC}{{\bf endcase}}
\newcommand{\TR}{{\bf true}}
\newcommand{\FA}{{\bf false}}
\newcommand{\T}{\hspace*{.3in}}

\def\ignore#1{\relax}
\providecommand{\email}[1]{\href{mailto:#1}{\nolinkurl{#1}\xspace}}

\begin{spacing}{3}
\noindent\textbf{Definitions:}
Define a problematic vertex as a vertex with degree greater than two. A vulnerable problematic vertex is a problematic vertex with at most one child that is not in a leaf path. An leaf subpath is a subpath which has one endpoint being a problematic vertex and one endpoint having degree one. An internal subpath is a subpath whose endpoints are both problematic vertices, or the root. We define the problem size $\Sigma$ as the number of vertices a feasibility test needs to check,
\[\Sigma=PV+l_1+2l_2+3l_4+4l_8+5l_{16}+\ldots,\]
where $PV$ is the number of unresolved problematic vertices and $l_{2^i}$ is the number of cleaned and glued paths whose length is between $2^{i-1}+1$ and $2^i$. The algorithm will maintain two selection lists, $L_1$ and $L_2$. The first selection list is for paths, while second selection list is for handling problematic vertices whose leaf paths have been resolved.
\vskip 0.2in\noindent
\textbf{Assumptions:}
For each problematic vertex $v$ with children vertices $c_1,\cdots, c_k$, where $k>2$, we create a complete binary tree rooted at $v$, whose bottom layer has $2^j\ge c_k>2^{j-1}$ vertices, including $c_1, c_2,\cdots, c_k$. All other vertices in the tree have vertex weight zero. Thus, we now have a binary tree while at most tripling the number of total vertices. Consequently, we can transform any given tree to a binary tree.
\vskip 0.2in\noindent
\textbf{Algorithm:}
Initialize the algorithm. Run Phase $1$ $l$ times, where $l$ is a constant which is later defined. Then run Phase $2$ a single time. While the weight inserted into $L_1$ at the end of Phase $2$ does not exceed the weight remaining in $L_1$ and the runtime on problematic vertices and resolved internal subpaths is at most half of the runtime, run Phase $1$ $l$ times and Phase $2$ a single time. When the runtime on problematic vertices and resolved internal subpaths is at least half of the runtime, repeatedly run Phase $2$ until there is no weight in $L_2$. When the weight insert into $L_1$ at the end of Phase $2$ exceeds the remaining weight in $L_1$, we continue the process, but use a different form of analysis.
\vskip 0.2in\noindent
\textbf{Initialization:}
Given a tree $T$ with $n$ vertices initially, take the edge-path-partition of $T$ and remove all problematic vertices. Take each subpath $P_i$ that remains and extend the length to $t_i$, the smallest possible power of two while labeling from the root, by adding in vertices with vertex weight zero, as necessary. For each subpath $P_i$, create an associated matrix, and insert its largest and smallest values into selection set $L_1$ with weight $\lceil\frac{4n^4}{t_i}\rceil$. For each subpath, repeatedly split the subpath, and insert the largest and smallest nonzero values of the associated matrices into $L_1$ with twice the weight of the previous weight. For each problematic vertex, insert paths of length of $2^i$ into $D$ with weight $\lceil\frac{4n^4}{2^i}\rceil$ while $t_i$ is less than two times the maximum distance of a vertex from the root.  Finally, for each internal subpath, extend the length to $2^i$, the smallest possible power of two longer than its length, and insert into $D$ with weight $\lceil\frac{4n^4}{2^j}\rceil$, where $2_j$ is the smallest possible power of two longer than $i$..
\vskip 0.2in\noindent
\textbf{Phase 1:} Select the weighted and unweighted medians from $L_1$, test for feasibility, and adjust $\lambda_1$ and $\lambda_2$ accordingly. During the feasibility testing, glue and clean adjacent subpaths which have been resolved. If any leaf path has been completely resolved, the leaf path can be represented by a single vertex.
\vskip 0.2in\noindent
\textbf{Phase 2:} For any vulnerable problematic vertex with a resolved leaf path, insert into $L_2$ the weight of that vertex plus the accumulated remaining weight from the resolved leaf path, for the max-min problem. Select the median from $L_2$, perform the feasibility test, and adjust $\lambda_1$ and $\lambda_2$ accordingly. In the max-min problem, if any vulnerable problematic vertices are resolved in this phase, we transfer the corresponding subpaths into $L_1$ from $D$ with the appropriate weight. Furthermore, if any internal path becomes a leaf path following the feasibility test, we remove the corresponding weight from $D$.


\newpage\noindent
\begin{theorem}
When the weight inserted by $L_2$ into $L_1$ is less than the weight in $L_1$, running the selection on $L_1$ five times reduces the weight remaining in $L_1$ by a factor of $1/6$.
\end{theorem}
\begin{proof}
By Lemma 3.2 of the path partition paper, each iteration decreases the weight remaining in $L_1$ by a factor of at least $1/6$. Suppose at the end of iteration $i$, the weight remaining in $L_1$ is $W$. Since $(5/6)^4<1/2$, then selecting the median of $L_1$ and running the feasibility test five times leaves $L_1$ with at most $(1/2)(5/6)W$ weight. Since the weight inserted by $L_2$ is less than the remaining weight in $L_1$, the weight in $L_1$ is at most $(5/6)W$ at the end of the iteration.
\end{proof}
\begin{theorem}
When the weight inserted by $L_2$ into $L_1$ is less than the weight in $L_1$, the $i$-th iteration of the algorithm will spend amortized time $O(i(5/6)^{i/5}n)$ on the unresolved internal subpaths and the leaf paths.
\end{theorem}
\begin{proof}
By similar reasoning to Lemma 3.2, each $1\times1$ submatrix quartered repeatedly from a matrix representing a path of length $2^j$ has weight $n^4/2^{4j-5}$. By iteration $i$, where $i$ satisfies $(4/3)*4n^5*(5/6)^i=n^2/2^{5j-5}$ or $2^{5j}=6*(6/5)^i$. However, at most $1/2^{5k}$ of the subpaths of length $2^{j-k}$ can be unresolved and at most $(1/2^{5j-2})$ of the subpaths of length $1$ can be unresolved. By Lemma 3.1, each subpath can be searched in amortized time proportional to its length, so the time to search the leaf subpaths and unresolved internal subpaths on iteration $i$ is at most
\[(j/2^j)n(1+1/2^5+1/2^{10}+\cdots)=O((j/2^j)n).\]
Since $2^j=6^{1/5}(6/5)^{i/5}$, then $j=(1/5)\log 6+(i/5)\log(6/5)$.
\end{proof}
\begin{corollary}
When the runtime on problematic vertices and resolved internal subpaths is at most half of the runtime, the $i$-th iteration of the algorithm will spend amortized time $O(i(5/6)^{i/5}n)$ on the unresolved internal subpaths and the leaf paths.
\end{corollary}

\begin{theorem}
When the weight inserted by $L_2$ into $L_1$ is less than the weight in $L_1$ and no leaf paths are completely resolved, it is not possible for the runtime on problematic vertices and resolved internal subpaths to be at most half of the runtime.
\end{theorem}
\begin{proof}
Suppose on iteration $i$, where $i$ satisfies $(4/3)*4n^5*(5/6)^i=n^2/2^{5j-5}$, there exists $l$ such that there are $N$ resolved internal subpaths of length $2^{j-l}$, where $N>(1/2^{5k})\left(\frac{n}{2^{l-k}}\right)$. Since there are no resolved leaf paths, there must be $N$ unresolved paths of average length $j-l$ for the runtime on problematic vertices and resolved internal subpaths to be at most half of the runtime. However, since $j-l<2^{j-l}$, the previous theorem guarantees there cannot be $N$ unresolved paths of average length $j-l$.
\end{proof}

\begin{claim}
When the weight inserted by $L_2$ into $L_1$ is at least the weight in $L_1$, the amortized time for resolving and releasing $m$ problematic vertices is better than the amortized time for resolving and releasing $1$ problematic vertex $m$ times.
\end{claim}

\begin{claim}
When a single problematic vertex is resolved and released and the weight inserted by $L_2$ into $L_1$ is at least the weight in $L_1$, the runtime for feasibility testing is $O(c+\log^2n)$, where $c$ is the number of remaining problematic vertices.
\end{claim}

\begin{claim}
The amortized time for resolving all resolved leaf paths in Phase $2$ is $O(n)$.
\end{claim}
\end{spacing}
\end{document}