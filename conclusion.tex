\section{Conclusion}
Our algorithms solve the max-min and min-max path $k$-partition problems, as well as the max-min tree $k$-partition problem in asymptotically optimal time. 
Consequently, they use the paradigm of parametric search without incurring any $\log$ factors. 
We avoid these $\log$ factors by searching within collections of candidate values that are implicitly encoded as matrices or as single values. 
We assign synthetic weights for these values based on their corresponding path lengths so that weighted selections favor the early resolution of shorter paths, which will speed up subsequent feasibility tests. 
In fact, our analysis relies on demonstrating a constant fraction reduction in the feasibility test time as the algorithm progresses. 
Simultaneously, unweighted selections quickly reduce the overall size of the set of candidate values so that selecting test values is also achieved in linear time. 

Furthermore, we have successfully addressed the challenge of developing a meaningful quantity to track progress as the algorithm proceeds. 
We have proved that the time to perform a feasibility test describes in a natural way the overall progress from the beginning of the algorithm. 
Unfortunately, even these observations alone are not enough to overcome the challenges of tree partitioning. 
In particular, without both compressing long paths and pruning leaf paths quickly, the feasibility test time might improve too slowly for the overall algorithm to take linear time. 
Our dual-pronged strategy addresses both issues simultaneously, demonstrating that parallel algorithms are not essential or even helpful in designing optimal algorithms for certain parametric search problems.